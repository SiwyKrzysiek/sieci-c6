\documentclass{article} % Polska wersja klasy article

\usepackage{polski} % Pozwala na użycie polskiego. Ustawia między innymi fontenc na T1
\usepackage[utf8]{inputenc} % Informuje o kodowaniu
\usepackage{textcomp} % Znaki specjalne takie jak ~
\usepackage{xcolor} % Definicje kolorów
\usepackage{tcolorbox} % Ładne ramki

\renewcommand{\labelitemi}{\textbullet} % Zmiana symbolu wliczeń

\usepackage{graphicx}
\graphicspath{ {./Obrazy/} }
\usepackage{subcaption} % Subfigury
\usepackage{float} % Pozycjonowanie figur
\usepackage{mwe} % Tymczasowe grafiki

\usepackage{listings} % Listingi kodu
\lstset{
  basicstyle=\ttfamily,
  showstringspaces=false,
  commentstyle=\color{gray},
  keywordstyle=\color{blue},
  breaklines=true,
  % frame=L,
  language=bash
}

\newcommand{\paragraphnl}[1]{\paragraph{#1} \mbox{} \\} % Paragraf z nową linią

\title{Laboratorium sieci komputerowych - c6 \\ Docker}
\author{Krzysztof Dąbrowski gr. 3}
\date{\today}

\begin{document}
\maketitle{}
\tableofcontents{}
%\newpage

\section{Cel zajęć}
Celem laboratorium jest uruchomienie kontenera udostępniającego serwis sieciowy przy pomocy programu \textit{Docker}.

Pracowałem na maszynie laboratoryjnej \texttt{c3}, na której był uruchomiony system CoreOS.

\section{Kontener testowy}
Na początek wyświetliłem wersję Dockera by upewnić się, że działa.

\begin{tcolorbox}[colback=yellow!10!white,colframe=red!45!black,coltitle=yellow!100!black, title=Sprawdzienie czy Docker działa]
  \begin{lstlisting}
  docker version
  \end{lstlisting}
  \tcblower
  \footnotesize
  \begin{lstlisting}
    Client: 
    Version:           18.06.3-ce
    API version:       1.38
    Go version:        go1.10.8
    Git commit:        d7080c1
    Built:             Tue Feb 19 23:07:53 2019
    OS/Arch:           linux/amd64
    Experimental:      false
   
   Server:
    Engine:
     Version:          18.06.3-ce
     API version:      1.38 (minimum version 1.12)
     Go version:       go1.10.8
     Git commit:       d7080c1
     Built:            Tue Feb 19 23:07:53 2019
     OS/Arch:          linux/amd64
     Experimental:     false 
  \end{lstlisting}
\end{tcolorbox}
\normalsize
\vspace{5mm}

\noindent{}Pobrałem obraz \texttt{hello-world} poleceniem
\begin{center}
  \texttt{docker pull hello-world}
\end{center}
Następnie uruchomiłem kontener na podstawie pobranego obrazu.

\begin{tcolorbox}[colback=yellow!10!white,colframe=red!45!black,coltitle=yellow!100!black, title=hello-world]
  \begin{lstlisting}
  docker run hello-world
  \end{lstlisting}
  \tcblower
  \scriptsize
  \begin{lstlisting}
  Hello from Docker!
  This message shows that your installation appears to be working correctly.

  To generate this message, Docker took the following steps:
  1. The Docker client contacted the Docker daemon.
  2. The Docker daemon pulled the "hello-world" image from the Docker Hub.
      (amd64)
  3. The Docker daemon created a new container from that image which runs the
      executable that produces the output you are currently reading.
  4. The Docker daemon streamed that output to the Docker client, which sent it
      to your terminal.
  \end{lstlisting}
\end{tcolorbox}
\normalsize

\section{Kontener z serwerem WWW}
Wybrałem kontener \texttt{httpd} w wersji 2.4 ponieważ jest lekki i posiada skonfigurowany serwer \textit{Apache}.

Utworzyłem folder \texttt{WWW} ma maszynie \texttt{c3} oraz umieściłem w pliki index.html, script.js i style.css prostej strony internetowej, którą napisałem w wolnym czasie.

Pobrałem obraz poleceniem
\begin{center}
  \texttt{docker pull httpd:2.4}
\end{center}
Następnie uruchomiłem kontener.
\begin{center}
  \scriptsize
  \texttt{docker run -d -p 8080:80 -v /home/stud/www:/usr/local/apache2/htdocs/ httpd:2.4}
  \normalsize
\end{center}

Zastosowałem następujące flagi
\begin{itemize}
  \item \texttt{d} -- Uruchomienie kontenera w tle ,
  \item \texttt{p} -- Mapowanie portów maszyny na porty kontenera,
  \item \texttt{v} -- Przekazanie katalogów maszyny kontenerowi.
\end{itemize}

\end{document}